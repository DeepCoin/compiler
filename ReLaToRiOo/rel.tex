
\documentclass[12pt]{article}
\usepackage[utf8]{inputenc}
\usepackage{graphicx}
\usepackage{amsmath}
\usepackage[margin=1in]{geometry}
\usepackage{indentfirst}
\usepackage{amsfonts}
\usepackage[portuguese]{babel}
\usepackage{float}
\usepackage[usenames,dvipsnames]{color}







\begin{document}

\begin{titlepage}

\newcommand{\HRule}{\rule{\linewidth}{0.5mm}} 
\center 
 

\textsc{\LARGE universidade de uoimbra}\\[1.5cm] % Name of your university/college
\textsc{\Large departamento de engenharia informática}\\[4cm] % Major heading such as course name
\textsc{\large compiladores}\\[1cm] % Minor heading such as course title


\HRule \\[0.5cm]
{ \huge \bfseries Compilador para a linguagem iJava}\\[0.4cm] 
\HRule \\[8cm]
 
\begin{minipage}{0.4\textwidth}
\begin{flushleft} \large
\emph{Autor:}\\
Bruno \textsc{Caceiro}  \\caceiro@student.dei.uc.pt
\end{flushleft}
\end{minipage}
~
\begin{minipage}{0.4\textwidth}
\begin{flushright} \large
\emph{Autor:} \\
David \textsc{Cardoso}  \\davidfpc@student.dei.uc.pt
\end{flushright}
\end{minipage}\\[2cm]



{\large \today}\\[3cm]

\vfill

\end{titlepage}



\section{Introdução}
Este projecto consiste no desenvolvimento de um compilador para a linguagem \emph{iJava} (imperative Java), que consiste num pequeno subconjunto da linguagem Java (versão 5.0). Os programas da linguagem \emph{iJava} são constituídos por uma única classe (a principal), contendo necessariamente um método \emph{main}, e podendo conter outros métodos e atributos, todos eles estáticos e (possivelmente) públicos.

O projecto foi estruturado em 3 fases, primeiramente foi feita a Análise Lexical, implementada na linguagem \emph{C} e utilizando a ferramenta \emph{lex}. A segunda fase consistiu na análise sintática, com a  construção da árvore de sintaxe abstrata e análise semântica (tabelas de símbolos, deteção de erros semânticos). No final foi feita a geração de código.



	
	


\end{document}